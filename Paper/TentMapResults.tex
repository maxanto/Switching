\subsubsection{TENT} \label{sssec:tent}

When this map is implemented in a computer using any numerical representation system (even floating-point!) truncation errors rapidly increase and make unstable fixed point in $x^*=0$ to become stable.
The sequences within the attractor domain of this fixed point will have a short transitory of length between $0$ and $B$ followed by an infinite number of $0$'s \cite{Jessa2002,Callegari}.
This issue is easily explained in \cite{Li2004}, the problem appears because all iterations have a left-shift operation that carries the $0$'s from the right side of the number to the most significant positions.

Figs. \ref{fig:Hval_Tent} to \ref{fig:MP_Tent} show the quantifiers for floating- and fixed-point numerical representations.
$H_{hist}$, $H_{BP}$ and $C_{BP}$ quantifiers are equal to zero for all precisions, this reflects that the series quickly converge toward a fixed point for every initial conditions.
In the case of $H_{BPW}$ and $C_{BPW}$ quantifiers are different to zero because BPW procedure discards the elements once the fixed point is reached.
The high dispersions in $H_{BPW}$, $C_{BPW}$ and MP are related to the short length of serie's transient.
These transients that converge to a fixed point have a maximum length of $B$ elements (iterations) for fixed-point arithmetic and $80$ for floating-point (long double precision).

\begin{figure}[H]
	\centering
	\begin{subfigure}[b]{0.49\textwidth}
		\includegraphics[width=\textwidth]{Hval_Tent}
		\caption{$H_{hist}$ vs. $B$}
		\label{fig:Hval_Tent}
	\end{subfigure}
	\begin{subfigure}[b]{0.49\textwidth}
		\includegraphics[width=\textwidth]{Hbp_Tent}
		\caption{$H_{BP}$ vs. $B$}
		\label{fig:Hbp_Tent}
	\end{subfigure}
	\begin{subfigure}[b]{0.49\textwidth}
		\includegraphics[width=\textwidth]{Hbpw_Tent}
		\caption{$H_{BPW}$ vs. $B$}
		\label{fig:Hbpw_Tent}
	\end{subfigure}
	\begin{subfigure}[b]{0.49\textwidth}
		\includegraphics[width=\textwidth]{Cbp_Tent}
		\caption{$C_{BP}$ vs. $B$}
		\label{fig:Cbp_Tent}
	\end{subfigure}
	\begin{subfigure}[b]{0.49\textwidth}
		\includegraphics[width=\textwidth]{Cbpw_Tent}
		\caption{$C_{BPW}$ vs. $B$}
		\label{fig:Cbpw_Tent}
	\end{subfigure}
	\begin{subfigure}[b]{0.49\textwidth}
		\includegraphics[width=\textwidth]{MP_Tent}
		\caption{MP vs. $B$}
		\label{fig:MP_Tent}
	\end{subfigure}
	\caption{Statistical properties of TENT map.}
	\label{fig:TENT_QuantiB}
\end{figure}

Summarizing, in spite of using a high number of bits (with any 2-based numerical representation) to represent the digitalized TENT map it always converges quickly to the fixed point in $(x_n, x_{n+1})=(0, 0)$.
