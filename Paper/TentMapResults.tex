\subsubsection{Tent map (TENT)} \label{sssec:tent}

The Tent map has been extensively studied in the literature because theoretically it has nice statistical properties that can be analytically obtained.
For example it is easy to proof that it has a uniform histogram and consequently an ideal $H_{val}=1$.
The Perron-Frobenius operator and its corresponding eigenvalues and eigenfunctions may be also be analytically obtained for this map \cite{tent}.
\textcolor{red}{PONER ALGUNA REFERNCIA A PERRON Y VER SI ESTO VA ACÁ O EN ITQS}

This map is represented with the equation:
\begin{equation}\label{eq:tentmap}
x_{n+1}~=~ \left\{ \begin{array}{ll}
2~{x_n} & \textrm{if ~$0\leq x_n\leq 1/2$}\\
2~(1-{x_n}) & \textrm{if ~$1/2<x_n\leq 1$} 
\end{array} \right.  \ ,
\end{equation}
with $x_n\in\mathcal{R}$.
In base-2 fractional numbers rounding, equation \ref{eq:tentmap} becomes:
\begin{equation}\label{eq:tentdecbin}
x_{n+1}~=~ \left\{ \begin{array}{ll}
2~{x_n} & \textrm{if $0\leq x_n\leq 1/2$}\\
\epsilon \times floor\{\frac{2~-~2~x_n}{\epsilon}\} & \textrm{if $1/2<x_n\leq 1$} 
\end{array} \right.  \ ,
\end{equation}
with $\epsilon=2^{-B}$.

When this map is implemented in a computer using any numerical representation system (even floating point!) truncation errors rapidly increase and make unstable fixed point in $x^*=0$ to become stable.
The sequences within the attractor domain of this fixed point will have a short transitory of length between $0$ and $B$ followed by an infinite number of  $0$'s \cite{Jessa1993,Callegari1997}.
This issue is easily explained in [with chaos meet computers], the problem appears because all iterations have a left-shift operation that carries the $0$'s from the right side of the number.
Some authors \cite{buscar} have proposed to add a random perturbations to avoid this drawback of the Tent map.
This procedure improves the statistical properties of the time series, but what really happens is that the statistical properties of the random perturbations are mixed with those of the Tent map.
Here we study the Tent map ``as it is" without any artifact to evaluate its real instead of theoretical statistical properties. 

Figs. \ref{fig:TENT_QuantiB} (a) to (e) show the quantifiers for floating and fixed point numerical representations.
Quantifiers $H_{val}$, $H_{BP}$ and $C_{BP}$ are equal to zero for all prepositions, this reflects that the series quickly converge toward a fixed point for almost all sequences.
In the case of $H_{BPW}$ and $C_{BPW}$ quantifiers they are different of $0$ because $BPW$ procedure discards the elements once they reach the fixed point.
The high dispersions in $H_{BPW}$, $C_{BPW}$ and $MP$ are related with the short length of transient.
These transient to fixed point has a maximum length of $B$ iterations for fixed point arithmetic and $54$ for floating point (double precision).

\begin{figure}
	\includegraphics[width=.49\textwidth]{Hval_Tent}
	\includegraphics[width=.49\textwidth]{Hbp_Tent}
	\includegraphics[width=.49\textwidth]{Hbpw_Tent}
	\includegraphics[width=.49\textwidth]{Cbp_Tent}
	\includegraphics[width=.49\textwidth]{Cbpw_Tent}
	\includegraphics[width=.49\textwidth]{MP_Tent}
	\caption{Statistical properties of the TENT map: (a) $H_{val}$ vs $B$ (b) $H_{BP}$ vs $B$ (c) $C_{BP}$ vs $B$ (d) $MP$ vs $B$.}
	\label{fig:TENT_QuantiB}
\end{figure}

In summary in spite of using a high number of bits (with any 2-based numerical representation) to represent the digitalized TENT map it always loses the chaotic behaviour.
