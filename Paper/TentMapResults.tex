\subsubsection{Tent map (TENT)} \label{sssec:tent}

\begin{equation}\label{eq:tentmap}
x_{n+1}~=~ \left\{ \begin{array}{ll}
2~{x_n} & \textrm{if ~$0\leq x_n\leq 1/2$}\\
2~(1-{x_n}) & \textrm{if ~$1/2<x_n\leq 1$} 
\end{array} \right.  \ ,
\end{equation}
with $x_n\in\mathcal{R}$.
%
The Tent map has been extensively studied in the literature because theoretically it has nice  statistical properties that can be analytically obtained. For example it is easy to proof that it has a uniform histogram and consequently an ideal $H_{hist}=1$. The Perron-Frobenius operator and its corresponding eigenvalues and eigenfunctions may be also be analytically obtained for this map \cite{tent}. 

When this map is implemented in a computer using any numerical representation system (even floating point!) truncation errors rapidly increases and makes the unstable fixed point in $x^*=0$ becomes stable producing a short transitory followed by an infinite number of  $0$'s\cite{Jessa1993,Callegari1997}. Some authors \cite{buscar} have proposed to add a random perturbation to avoid this drawback of the Tent map. But this procedure introduces statistical properties of the random perturbation that are mixed with those of the Tent map itself.

Here we study the Tent map ``as it is«« without any artifact to evaluate its real instead of theoretical statistical properties. Note that to effectively work in a given representation it is necessary to change the expression of the map in order to make all the operations in the chosen representation numbers. For example, in the case of TENT the expression in decimal numbers is:

\begin{equation}\label{eq:tentdecbin}
x_{n+1}~=~ \left\{ \begin{array}{ll}
2~{x_n} & \textrm{if $0\leq x_n\leq 1/2$}\\
\epsilon \times floor\{\frac{2~-~2~x_n}{\epsilon}\} & \textrm{if $1/2<x_n\leq 1$} 
\end{array} \right.  \ ,
\end{equation}
with $\epsilon=10^{-P}$ for decimal numbers and $\epsilon=2^{-B}$ for binary numbers. In Eq. \ref{eq:tentdecbin} $x_n$ is either a decimal number with $P$ digits or a binary number with $B$ bits.

Figs. \ref{fig:tent} (a) to (e) show the different quantifiers for floating point and decimal numerical representation. In each figure from (a) to (c) a dashed line shows the value for the floating point representation. In figures (d) and (e) the star corresponds to the floating point case. In decimal representations the value of $H_{hist}$ remains almost constant for $11\leq P\leq 16$ (see Fig. \ref{fig:tent} (a)). Its value is $<H_{hist}>=0.8740$ with a variance $\sigma_{Hhist}=2.5 \times 10^{_6}$. For lower or higher values pf $P$ entropy decreases. This effect is due precisely to the stabilization of the fixed point at $x=0$. For ordering patterns entropy $H_{BP}$ an almost constant value is obtained for $8\leq P \leq 15$. The value is  $H_{BP}\simeq 0.6287$ with variance $\sigma_{H_{BP}}=4.8 \times 10^{-6}$ (see Fig. \ref{fig:tent} (b)). This rather small maximum value may be understood by seeing  Fig. \ref{fig:tent} (c), where the number of MP.
% 
is minimal for $P$ within the same range but it is still large: $645$ patterns are missing and only $75$ ordering patterns are present in the time series. Then, even with a uniform distribution between these $75$ patterns, entropy can not be higher than $ln(75)/ln(720)\simeq 0.65$. 
A more complete perspective of the statistical properties is obtained in Fig. \ref{fig:tent} (d) showing the representative point in the $H_{hist},H_{BP}$ plane for different precisions. Note that the best choice for maximum stochasticity is obtained for $11\leq P \leq 15$, with maximum attainable values for both entropies.
Increasing the number of decimal figures makes Tent map worst in the sense the system approaches the state for the floating point representation (the star at $(0,0)$. 
Statistical complexity $C_{BP}$ is also maximal for $8\leq P \leq 15$. 
Fig. \ref{fig:tent} (e) shows the representation on the $H_{BP},C_{BP}$ plane. In this plane it is also clear that the more stochastic option corresponds with  $11\leq P \leq 15$ but even in the optimum case the representative point is located in a position very similar to other chaotic maps, very far from the ideal point for stochastic systems in this plane that is $(1,0)$ \cite{Rosso2007C}.
Binary numerical representation of the Tent map remains very near to the floating point values for  $1 \leq B \leq 27 $ (see Fig. \ref{fig:tent} (f)).
The conclusion is it is convenient to use a decimal numbers representation with  $P=11$ to get the optimum time series for the Tent map. A higher number of decimal figures does not improve the statistical properties of the time series. Furthermore binary and floating point representations are not allowed. 

\begin{figure}
	\includegraphics[width=.32\textwidth]{Hval_Tent}
	\includegraphics[width=.32\textwidth]{Hbp_Tent}
	\includegraphics[width=.32\textwidth]{Hbpw_Tent}
	\includegraphics[width=.32\textwidth]{Cbp_Tent}
	\includegraphics[width=.32\textwidth]{Cbpw_Tent}
	\includegraphics[width=.32\textwidth]{MP_Tent}
	\includegraphics[width=.32\textwidth]{Period_Tent}
	\includegraphics[width=.32\textwidth]{HbpHval_Tent}
	\includegraphics[width=.32\textwidth]{HbpwHval_Tent}
	\includegraphics[width=.32\textwidth]{CbpHbp_Tent}
	\includegraphics[width=.32\textwidth]{CbpwHbpw_Tent}
	\caption{Statistical properties of the SWITCH map using binary representation: (a) $H_{hist}$ vs $P$ (b) $H_{BP}$ vs $P$ (c) $C_{BP}$ vs $P$ (d) Number of missing ordering patterns $MP$ vs $P$. In Figures (a) to (d) dashed line correspond to floating point numbers. (e) representation in the $H_{hist},H_{BP}$ plane in the the binary numerical system.  The star represents the state for floating points numbers. (f) representation in the $H_{BP},C_{BP}$ plane.  The star represents the state for floating points numbers. (The star represents the state for floating points numbers). } \label{fig:seqbin}
\end{figure}
