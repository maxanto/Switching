\section{Conclusions}
\label{sec:conclusions}

In this paper we explored the statistical degradation of simple, switched and skipped chaotic maps due to the inherent error of a based-2 systems.
We evaluated mixing and amplitude distributions from a statistical point of view.

Our work complements previous results given in \cite{Nagaraj2008}, where period lengths were investigated.
In that sense, our results were compatible with these.
We can see that the switching between two maps increases the dependence of period as function of precision, nevertheless the standard procedure of skipping reduce the period length in almost a half.

All statistics of the maps represented in fixed-point produces a non-monotonous evolution toward the floating-point results.
This result is relevant because it shows that increasing the precision is not always recommended.

Our results show that SWITCH has a marginal improvement in the mixing with respect to LOG and TENT.
However the greatest improvement comes when skipping is applied, we can see that BP and BPW entropies grow and BP and BPW complexities decrease, for the same numerical representation.
This result is relevant because evidences that a long period is not synonymous of good statistics, switched maps EVEN and ODD have half period lengths but their mixing is better and their amplitude distributions remain almost equal.
As counterpart, more precision is needed to reach the better asymptotes that offers the switching method.

\textcolor{red}{It was especially interesting to note that the TENT maps with $u = 2$ (which produces outputs that quickly converge to zero) and $u = 1.96$ (with statistical properties better than LOG), produce outputs with the same results when they are included in the switched scheme.}