\section{Conclusions}\label{sec:conclusions}
In this paper we explore the statistical degradation of simple, switched and skipped chaotic maps due to the inherent error of a based-2 systems.
We evaluate mixing and amplitude distributions from a statistical point of view.

Our work complements the previous results given in \cite{Nagaraj2008}, where period was investigated.
In that sense, our results were compatible with these.
We can see that the switching between two maps increase the dependence of period as function of precision, this is because the correlation length is also increased.
Neverthless, the standard procedure of skipping reduce the period length in almost a half.

All statistics of the maps represented in fixed-point produces a non-monotonous evolution toward the floating-point results.
This result is relevant because it shows that increasing the precision is not always recommended.

It is specially interesting to note that some systems (TENT) with very nice statistical properties in the world of the real numbers, become ``pathological" when binary numerical representations are used.
As a rule, if we only needs shift operations to calculate a map (it depends on the base of the arithmetic logic unit and the map itself), all initial conditions converges to a fixed point with a transient no longer than the length of their mantissa.

By comparing between BP and BPW quantifiers, we will able to detect falls to a fixed point and estimate the relative length of transient. It can be seen in all implementations of TENT and in one initial condition of SWITCH and EVEN for floating-point implementation.

Related to statistical behaviour, our results shows that SWITCH has a marginal improvement in the mixing with respect to LOG (and TENT, of course).
However the great improvement comes when skipping is applied, we can see that BP and BPW entropies grow and BPW and BPW complexities decrease, for the same numerical representation.
