\section{Results}\label{sec:resultados}

Five pseudo-chaotic maps were studied, two simple maps and three combination of them.
For each one we have used numbers represented by floating-point (double presicion 754-IEEE standard) and  fixed-point numbers with $1\leq B \leq 53$, where $B$ is the number of bits that represents the fractional part.
Time series were generated using $100$ randomly chosen initial conditions within their attraction domain (interval $[0,1]$), for each one of these $54$ number precisions.
The studied maps are logistic (LOG), tent (TENT), sequential switching between TENT and LOG (SWITCH) and skipping discarding the values in the odd positions (EVEN) or the values in the even positions (ODD) respectively.
\textcolor{red}{In the following, all the results will be obtained from the pseudo-chaotic version of the LOG, TENT, SWITCH, EVEN and ODD maps.}

Logistic map is interesting because it is representative of the very large family of quadratic maps.
Its expression is:
%
\begin{equation}\label{eq:LOG}
x_{n+1}=4~x_{n}~(1-x_{n}) 
\end{equation}
%
with $x_n \in \mathbb{R}$.

Note that to effectively work in a given representation it is necessary to change the expression of the map in order to make all the operations in the chosen representation numbers. For example, in the case of LOG the expression in binary fixed-point numbers is:
%
\begin{equation}\label{eq:LOGB2}
x_{n+1}=4 ~\epsilon ~\text{floor}\left\{\frac{x_n(1-x_n)}{\epsilon}\right\}
\end{equation}
%
with $\epsilon = 2^{-B}$ where $B$ is the number of bits that represents the fractional part.

\textcolor{red}{This rounding technique is the same as that used in \cite{Antonelli2012, Grebogi1988, Nagaraj2008} and has some advantages,such as, it is algorithmically easy to implement and is independent of the platform where it is used, as long as $B$ is lower than the mantissa of the arithmetic of the local machine.
In our case, the results were obtained with an Intel I7 PC, which has an ALU with IEEE-754 floating point standard with double precision, which limits the method to $B \leq 53$ bits.}

Floating point representation does not constitute a field, wherein the basic maths properties, such as distributive, associative, are preserved.
However, in fixed point arithmetics the exponent does not shift the point position and some properties as distributive remains.
Also, we adopt the floor rounding mode because it arises naturally in a FPGA implementation.

The TENT map has been extensively studied in the literature because theoretically it has nice statistical properties that can be analytically obtained.
For example it is easy to proof that it has a uniform histogram and consequently an ideal $H_{hist}=1$.
The Perron-Frobenius operator and its corresponding eigenvalues and eigenfunctions may be analytically obtained for this map \cite{Lasota1994}.

Tent map is represented with the equation:
%
\begin{equation}\label{eq:TENT}
x_{n+1} =
\begin{cases}
	u~x_n &, \textrm{if } 0\leq x_n\leq 1/u\\
	\frac{u}{1-u}~(1-x_n) &, \textrm{if } 1/u< x_n\leq 1 
\end{cases}
\end{equation}
%
\textcolor{red}{with $x_n and $u$ \in \mathbb{R}$}.

In base-2 fractional numbers rounding, equation (\ref{eq:TENT}) becomes:
%
\begin{equation}\label{eq:TENTB2}
x_{n+1} = 
\begin{cases}
	\epsilon ~\text{floor} \{\frac{1}{\epsilon} \mu~(x_n)\} &, \textrm{if } 0\leq x_n\leq \mu^-\\
	\epsilon ~\text{floor} \{\frac{1}{\epsilon} \rho~(1-x_n)\} &, \textrm{if } \mu^-<x_n\leq 1 
\end{cases}	
\end{equation}
%
with $\epsilon=2^{-B}$, $\mu = \epsilon ~\text{floor}\{\frac{1}{\epsilon} u\}$, $\mu^- = \epsilon ~\text{floor}\{\frac{1}{\epsilon} (1/\mu)\}$ and $\rho = \epsilon ~\text{floor}\{\frac{1}{\epsilon} ~(\mu/(1-\mu)) \}$.

\textcolor{red}{
In \cite{DelaFraga2017}, the authors showed the evolution of the entropy of values $H_{hist}$ with respect to the binary precision.
They characterized the evolution of the TENT map as a function of binary precision in a fixed-point arithmethics.
In their scheme of generation of random numbers they used two postprocessing stages, first they binarized the data by detecting the crossing by a threshold, and then these data were processed by a XOR gate.
In our case we use the output of the chaotic maps without any randomization process, however their results are very interesting to take a criterion about which parameters are useful to implement.}
\textcolor{red}{
In our case, we adopted two values of $u$ for two different reasons.
Following \cite{DelaFraga2017}, an interesting value is $ u = 1.96 $, or its closest value within the arithmetic used.
On the other hand, the value of $u = 2$ is very attractive due to its extremely low implementation cost.}

Switching, even and odd skipping procedures are shown in Fig. \ref{fig:seq}.
%
\begin{figure}[htpb]
\centering	
	\includegraphics[height=0.4\textheight]{SwitchParImpar}
	\caption{Sequential switching between Tent and Logistic maps. In the figure are also shown even and odd skipping strategies.}
	\label{fig:seq}
\end{figure}

SWITCH map is expressed as:
%
\begin{equation}\label{eq:SWITCH}
\begin{cases}
	x_{n+1}=
	\begin{cases}
		u~x_n &, \textrm{if } 0\leq x_n\leq 1/u\\
		\frac{u}{1-u}~(1-x_n) &, \textrm{if } 1/u< x_n\leq 1 
	\end{cases} \\
	x_{n+2}=4\,x_{n+1}\,(1-x_{n+1})
\end{cases}
\end{equation}
%
with $x_n \in \mathbb{R}$ and $n$ an even number.

\textcolor{red}{However, as in the previous cases, we work with its pseudo-chaotic counterpart that can be expressed as:}
%
\begin{equation}\label{eq:SWITCHB2}
\begin{cases}
	x_{n+1}=
	\begin{cases}
		\epsilon ~\text{floor} \{\frac{1}{\epsilon} \mu~(x_n)\} &, \textrm{if } 0\leq x_n\leq \mu^-\\
		\epsilon ~\text{floor} \{\frac{1}{\epsilon} \rho~(1-x_n)\} &, \textrm{if } \mu^-<x_n\leq 1
	\end{cases} \\
	x_{n+2}=4 ~\epsilon ~\text{floor}\left\{\frac{x_n(1-x_n)}{\epsilon}\right\}
\end{cases}
\end{equation}
%

Skipping is a usual randomizing technique that increases the mixing quality of a single map and correspondingly increases the value of $H_{BP}$ and decreases $C_{BP} $ of the time series.
Skipping does not change the values of $H_{hist}$ for ergodic maps because it are evaluated using the non causal PDF (normalized histogram of values) \cite{DeMicco2008}.

In the case under consideration we study even and odd skipping of the sequential switching of TENT and LOG maps:
\begin{enumerate}[leftmargin=*,labelsep=4.9mm]
	\item Even skipping of the sequential switching of Tent and Logistic maps (EVEN).\\
	If $\{x_n; n=1,\dots,\infty\}$ is the time series generated by eq. \eqref{eq:SWITCHB2}, discard all the values in odd positions and retain the values in even positions.
	\item Odd skipping of the sequential switching of Tent and Logistic maps.
	If $\{x_n; n=1,\dots,\infty\}$ is the time series generated by eq. \eqref{eq:SWITCHB2}, discard all the values in even positions and retain all the values in odd positions.
\end{enumerate}

Even skipping may be expressed as the composition function TENT$\circ$LOG while odd skipping may be expressed as LOG$\circ$TENT.
The evolution of period as function of precision was reported in \cite{Nagaraj2008} for these resulting maps.
\textcolor{red}{In this paper we use the same simulation algorithm and switching scheme as in \cite{Nagaraj2008} and we add the analysis from the statistical point of view.}

\subsection{Period as a function of binary precision}

Grebogi and coworkers \cite{Grebogi1988} have studied how the period $T$ is related with the precision.
There they saw that the period $T$ scales with roundoff $\epsilon$ as $T\sim\epsilon^{-d/2}$ where $d$ is the correlation dimension of the chaotic attractor.

Nagaraj \textit{et als}. \cite{Nagaraj2008} studied the case of switching between two maps.
They saw that the period $T$ of the compound map obtained by switching between two chaotic maps is higher than the period of each map and they found that a random switching improves the results.
Here we have considered sequential switching to avoid the use of another random variable, because it can include its own statistical properties in the time series.

Fig. \ref{fig:period} shows  $T$ vs. $B$ in semi logarithmic scale.
We run the attractor from $100$ randomly chosen initial conditions
The figure show: $100$ red points for each fixed-point precision ($1\geq B \geq 53$) and in black their average (dashed black line connecting black dots).
The experimental averaged points can be fitted by a straight line (in blue) expressed as $\log_10 T=m B + b$ where $m$ is the slope and $b$ is the $y$-intercept.
Results for the rest of considered maps are summarized in Table \ref{tabla:periodos}.

\begin{figure}[H]
\centering	
	\includegraphics[width=.49\textwidth]{Period_Log}
	\caption{Period as function of precision in binary digits (see text).}
	\label{fig:period}
\end{figure}

\begin{table}[H]
\centering	
	\caption{Period $T$ as a function of $B$ for the maps considered}
	\vspace{1em}
	\begin{tabular}{lll}
		\hline\noalign{\smallskip}
		map 	& m 	& b  \\
		\noalign{\smallskip}\hline\noalign{\smallskip}
		TENT	&0 		& 0 \\
		LOG 	&0.139 	& -0.6188 \\
		SWITCH 	&0.1462 & -0.5115 \\
		EVEN 	&0.1447 & -0.7783 \\
		ODD 	&0.1444 & -0.7683 \\
		\noalign{\smallskip}\hline
	\end{tabular}
	\label{tabla:periodos}	
\end{table}

Results are compatible for those obtained in \cite{Nagaraj2008}.
Switching between maps increases the period $T$ but skipping procedure decreases by almost half.

\subsection {Quantifiers of simple maps}
\label{subsec:SimpleMaps}
Here we report our results for both simple maps, LOG and TENT.

\subsubsection{Logistic map (LOG)} \label{subsubsec:log}

Logistic map is interesting because is representative of the very large family of quadratic maps.
Its expression is:
\begin{equation}\label{eq:logimap}
 x_{[n+1]}=4x_{[n]}(1-x_{[n]}) \,
\end{equation}
with $x_n\in\mathcal{R}$.

Note that to effectively work in a given representation it is necessary to change the expression of the map in order to make all the operations in the chosen representation numbers. For example, in the case of LOG the expression in binary fixed point numbers is:
\begin{equation}\label{eq:logimapB2}
x_{n+1}=4 \epsilon floor\{\frac{x_n(1-x_n)}{\epsilon}\} \,
\end{equation}
with $\epsilon = 2^B$ where $B$ is the number of bits that represents the fractional part.

According as B grows, statistical properties varies until they stabilize.
Figs. \ref{fig:LOG_QuantiB} (a) to (f) show the statistical properties of LOG map in floating point and fixed point representation.
All these figures show: $100$ red points for each fixed point precision ($B$) and in black their average (dashed black line connecting black dots), $100$ horizontal dashed blue lines that are the results of each run in floating point and a black solid line their average.
In this case, all the lines of the floating point are overlapped.

For $B\geq 30$ the value of $H_{val}$ remains almost identical to the values for the floating point representation whereas $H_{BP}$ and $C_{BP}$ stabilizes at $B>21$.
Their values are: $<H_{val}>=0.9669$; $<H_{BP}>=0.6269$; $<C_{BP}>=0.4843$.
Note that the stable value of missing patterns $MP=645$ makes the optimum $H_{BP} \leq ln(75)/ln(720) \simeq 0.65$.
Then $B=30$ is the most convenient choice because an increase in the number of fractional digits does not improve the statistical properties.

Some conclusions can be drawn regarding \textit{BP} and \textit{BPW} quantifiers.

For $B=1, 2, 3$ and $4$, the averaged $BP$ quantifiers are almost $0$ while the averaged $BPW$ quantifiers can not be calculated (seein Figs. \ref{fig:LOG_QuantiB} c and e the missing black dashed line).
For those sequences where the initial condition where $0$ all iterations result to be a sequence of $0$s (the fixed point of the map), this happens very likely with small presitions because of the roundoffs.

For $B=7, 9$ and $12$, a high dispersion in $BPW$ are showed, but not in $BP$ graphs, atractor falls to a fixed point with an short transitory in these cases.

\textcolor{red}{ESTA ORACIÓN VA A DESAPARECER CON LONG DOUBLE, O NO?}
Finally, for $B = 45, 49, 51$ and $52$ some $BP$ quantifiers have a low value, but $BPW$ quantifiers have a more predictible behavior, we can see that these atractors fall to a fixed point with a long transitory.
In this behaviour is due to the employed plataform.
The operation miltiplication needs the double of bits to be represented but the machine has only $64$.

\begin{figure}
	\includegraphics[width= .49\textwidth]{Hval_Logistico}
	\includegraphics[width= .49\textwidth]{Hbp_Logistico}
	\includegraphics[width= .49\textwidth]{Hbpw_Logistico}
	\includegraphics[width= .49\textwidth]{Cbp_Logistico}
	\includegraphics[width= .49\textwidth]{Cbpw_Logistico}
	\includegraphics[width= .49\textwidth]{MP_Logistico}
	\caption{Statistical properties of the LOG map using binary representation: (a) $H_{val}$ vs $B$ (b) $H_{BP}$ vs $B$ (c) $C_{BP}$ vs $B$ (d) Number of missing ordering patterns $MP$ vs $B$.}
	\label{fig:LOG_QuantiB}
\end{figure}

In Fig. \ref{fig:LOG_HH} double entropy planes are showed.
Red points are each individual measurement and black points their mean.
Black dashed line shows the evolution of the mean as $B$ growths.
Blue points shows each individual measurement in floating point and their mean with an star.

Although the distribution of values reaches high values their mixing is poor, this can be seen in the evolution of the mean values according $B$ growths.
Accross $B=20$, $H_{val}$ improves but $H_{BP}$ stay around its maximum.

\begin{figure}
	\includegraphics[width= .49\textwidth]{HbpHval_Logistico}
	\includegraphics[width= .49\textwidth]{HbpwHval_Logistico}
	\caption{Evolution of statistical properties in double entropy plane of LOG map using binary representation: (a) $H_{val}$ vs $H_{BP}$ (b) $H_{val}$ vs $H_{BPW}$.}
	\label{fig:LOG_HH}
\end{figure}

In Fig. \ref{fig:LOG_HC} we shows the entropy-complexity planes.
Dotted gray lines are the upper and lower margins, is spected that an chaotic system remain near upper margin.
There results characterize an chaotic behaviour, in $H_{BP}-C_{BP}$ plane we can see a low entropy and high complexity.

\begin{figure}
	\includegraphics[width= .49\textwidth]{CbpHbp_Logistico}
	\includegraphics[width= .49\textwidth]{CbpwHbpw_Logistico}
	\caption{Evolution of statistical properties in entropy-complexity plane of LOG map using binary representation: (a) $C_{BP}$ vs $H_{BP}$ (b) $C_{BPW}$ vs $H_{BPW}$.}
	\label{fig:LOG_HC}
\end{figure}


\subsubsection{TENT} \label{sssec:tent}

When this map is implemented in a computer using any numerical representation system (even floating-point!) truncation errors rapidly increase and make unstable fixed point in $x^*=0$ to become stable.
The sequences within the attractor domain of this fixed point will have a short transitory of length between $0$ and $B$ followed by an infinite number of $0$'s \cite{Jessa2002,Callegari}.
This issue is easily explained in \cite{Li2004}, the problem appears because all iterations have a left-shift operation that carries the $0$'s from the right side of the number to the most significant positions.
Some authors \cite{buscar} have proposed to add random perturbations to avoid this drawback of the Tent map.
This procedure improves the statistical properties of the time series, but what really happens is that the statistical properties of the random perturbations are mixed with those of the Tent map.
Here we study the Tent map ``as it is" without any artifice to evaluate its real behaviour, instead of theoretical statistical properties. 

Figs. \ref{fig:Hval_Tent} to \ref{fig:MP_Tent} show the quantifiers for floating- and fixed-point numerical representations.
Quantifiers $H_{hist}$, $H_{BP}$ and $C_{BP}$ are equal to zero for all precisions, this reflects that the series quickly converge toward a fixed point for all sequences.
In the case of $H_{BPW}$ and $C_{BPW}$ quantifiers are different to non-null because BPW procedure discards the elements once they reach the fixed point.
The high dispersions in $H_{BPW}$, $C_{BPW}$ and MP are related to the short length of series transient.
These transients that converge to a fixed point have a maximum length of $B$ iterations for fixed-point arithmetic and $80$ for floating-point (long double precision).

\begin{figure}[htpb]
	\centering
	\begin{subfigure}[b]{0.49\textwidth}
		\includegraphics[width=\textwidth]{Hval_Tent}
		\caption{$H_{hist}$ vs. $B$}
		\label{fig:Hval_Tent}
	\end{subfigure}
	\begin{subfigure}[b]{0.49\textwidth}
		\includegraphics[width=\textwidth]{Hbp_Tent}
		\caption{$H_{BP}$ vs. $B$}
		\label{fig:Hbp_Tent}
	\end{subfigure}
	\begin{subfigure}[b]{0.49\textwidth}
		\includegraphics[width=\textwidth]{Hbpw_Tent}
		\caption{$H_{BPW}$ vs. $B$}
		\label{fig:Hbpw_Tent}
	\end{subfigure}
	\begin{subfigure}[b]{0.49\textwidth}
		\includegraphics[width=\textwidth]{Cbp_Tent}
		\caption{$C_{BP}$ vs. $B$}
		\label{fig:Cbp_Tent}
	\end{subfigure}
	\begin{subfigure}[b]{0.49\textwidth}
		\includegraphics[width=\textwidth]{Cbpw_Tent}
		\caption{$C_{BPW}$ vs. $B$}
		\label{fig:Cbpw_Tent}
	\end{subfigure}
	\begin{subfigure}[b]{0.49\textwidth}
		\includegraphics[width=\textwidth]{MP_Tent}
		\caption{MP vs. $B$}
		\label{fig:MP_Tent}
	\end{subfigure}
	\caption{Statistical properties of TENT map}
	\label{fig:TENT_QuantiB}
\end{figure}

Summarizing, in spite of using a high number of bits (with any 2-based numerical representation) to represent the digitalized TENT map it always loses the chaotic behaviour.
 

\subsection{Quantifiers of combined Maps}\label{subsec:SecSwitch}
Here we report our results for the three combinations of the combined maps, SWITCH, EVEN and ODD.

\subsubsection{Sequential switching between Tent and Logistic maps (SWITCH)} \label{sssec:switch}

SWITCH may be expressed as a composition between $M_1 \circ M_2$ given by the following recurrence:
%
\[ \left\{ \begin{array}{ccc}\label{eq:seq}
x_{n+2}~=~ 4~x_{n+1}~(1-{n+1}) \\
x_{n+1}~=~ \left\{ \begin{array}{ll}
2~{x_n} & \textrm{if $0\leq x_n\leq 1/2$}\\
2~(1-{x_n}) & \textrm{if $1/2<x_n\leq 1$} 
\end{array} \right.  \end{array}\right. \] 
with $x_n\in\mathcal{R}$.
%
Results with sequential switching are shown in Figs. \ref{fig:seqbin} (a) to (f).
The floating point entropy value is $H_{hist}=0.8658$, a higher value to that obtained for LOG. 
For binary numbers this value is reached for $B=24$, but it is stable from $B=28$.
Regarding ordering patterns the number of MP decreases to $586$, a value lower than the one obtained for LOG.
It means the entropy $H_{BP}$ may increase up to $ln(134)/ln(720)\simeq 0.74$.
$BP$ and $BPW$ quantifiers reachs their maximum at $B=16$, but they stabilishes at $B=24$.
This means that an amount of $B \geq 28$ is necessary to obtain optimal results.
We can see some points with anomalies from $B \geq 49$ due to same problem of LOG map.

\begin{figure}
	\includegraphics[width=.32\textwidth]{Hval_Switch}
	\includegraphics[width=.32\textwidth]{Hbp_Switch}
	\includegraphics[width=.32\textwidth]{Hbpw_Switch}
	\includegraphics[width=.32\textwidth]{Cbp_Switch}
	\includegraphics[width=.32\textwidth]{Cbpw_Switch}
	\includegraphics[width=.32\textwidth]{MP_Switch}
	\includegraphics[width=.32\textwidth]{HbpHval_Switch}
	\includegraphics[width=.32\textwidth]{HbpwHval_Switch}
	\includegraphics[width=.32\textwidth]{CbpHbp_Switch}
	\includegraphics[width=.32\textwidth]{CbpwHbpw_Switch}
	\caption{Statistical properties of SWITCH,  using binary representation: (a) $H_{hist}$ vs $P$ (b) $H_{BP}$ vs $P$ (c) $C_{BP}$ vs $P$ (d) Number of missing ordering patterns $MP$ vs $P$. In Figures (a) to (d) dashed line correspond to floating point numbers. (e) representation in the $H_{hist},H_{BP}$ plane in the the binary numerical system.  The star represents the state for floating points numbers. (f) representation in the $H_{BP},C_{BP}$ plane.  The star represents the state for floating points numbers.  } \label{fig:seqbin}
\end{figure}

\subsubsection{Skipping on sequential switching between Tent and Logistic maps (EVEN and ODD)} \label{sssec:switch}

Skipping is a usual randomizing technique that increases the mixing quality of a single map and correspondingly increases the value of $H_{BP}$ and decreases $C_{BP} $ of the time series. Skipping does not change the values of $H_{hist}$ and $C_{hist}$ evaluated using the non causal PDF (normalized histogram)\cite{DeMicco2008}. In the case under consideration we study Even and Odd skipping of the sequential switching of Tent and Logistic maps.


\begin{enumerate}
	\item Even skipping of the sequential switching of Tent and Logistic maps (EVEN).\\
	If $\{x_n,~(n=1,...\infty)\}$ is the time series generated by \ref{eq:seq} discard all the values in odd positions and retain the values in even positions.
	\item Odd skipping of the sequential switching of Tent and Logistica maps.
	If $\{x_n,~(n=1,...\infty)\}$ is the time series generated by \ref{eq:seq} discard all the values in even positions and retain all the values in odd positions.
\end{enumerate}

\begin{figure}
	\includegraphics[height=0.4\textheight]{SwitchParImpar}
	\caption{Sequential switching between Tent and Logistic maps. In the figure are also shown even and odd skipping strategies} \label{fig:seq}
\end{figure}

The reason for studying even and odd skipping cases is the sequential switching map $M_{switch}$ is the composition of two different maps. Even skipping may be expressed as $M_{TENT}\circ M_{LOG}$ while odd skipping may be expressed as $M_{LOG}\circ M_{TENT}$.

This is very interesting to note that a great improvement is obtained using any of the skipping strategies but EVEN is slightly better than ODD.  

MP are reduced to $MP\simeq 163$ for EVEN and $MP\simeq 164$ for ODD, increasing the maximum allowed Bandt \& Pompe entropy that reaches the mean value $<H_{BP}>\simeq 0.905$ with variance $\sigma_{H_{BP}}\simeq=0.107 \times 10^{-6}$ for EVEN, and $<H_{BP}>\simeq 0.854$ with variance $\sigma_{H_{BP}}\simeq=0.285 \times 10^{-6}$ for a decimal representation with  $9\leq P\leq27$. The complexity is reduced to $<C_{BP}>\simeq 0.224$ with $\sigma_{C_{BP}}\simeq=0.166 \times 10^{-6}$ for EVEN and  $<C_{BP}>\simeq 0.282$ with $\sigma_{C_{BP}}\simeq=0.281 \times 10^{-6}$ for ODD.

Quantifiers related to the normalized histogram slightly degrades with the skipping procedure. For example $H_{hist}$ reduces from $0.866$ without skipping to $0.813$ for any EVEN or ODD. 

Results in binary numbers are similar to those obtained for the equivalent number of figures in decimal numbers. For example the minimum in MP is reached for $B=27$, and this number of bits is almost equivalent to $P=9$. 

In Figs. \ref{fig:seqpardec} and Figs. \ref{fig:seqparbin} are shown the results for EVEN. We do not give the Figs. for ODD because they are very similar, as pointed above.

\begin{figure}
	\includegraphics[width=.32\textwidth]{Hval_SwitchEven}
	\includegraphics[width=.32\textwidth]{Hbp_SwitchEven}
	\includegraphics[width=.32\textwidth]{Hbpw_SwitchEven}
	\includegraphics[width=.32\textwidth]{Cbp_SwitchEven}
	\includegraphics[width=.32\textwidth]{Cbpw_SwitchEven}
	\includegraphics[width=.32\textwidth]{MP_SwitchEven}
	\includegraphics[width=.32\textwidth]{Period_SwitchEven}
	\includegraphics[width=.32\textwidth]{HbpHval_SwitchEven}
	\includegraphics[width=.32\textwidth]{HbpwHval_SwitchEven}
	\includegraphics[width=.32\textwidth]{CbpHbp_SwitchEven}
	\includegraphics[width=.32\textwidth]{CbpwHbpw_SwitchEven}
	\caption{Statistical properties of EVEN, obtained by skipping the values in the odd position of the time series of  SWITCH,  using binary representation: (a) $H_{hist}$ vs $P$ (b) $H_{BP}$ vs $P$ (c) $C_{BP}$ vs $P$ (d) Number of missing ordering patterns $MP$ vs $P$. In Figures (a) to (d) dashed line correspond to floating point numbers. (e) representation in the $H_{hist},H_{BP}$ plane in the the binary numerical system.  The star represents the state for floating points numbers. (f) representation in the $H_{BP},C_{BP}$ plane.  The star represents the state for floating points numbers.  } \label{fig:seqimparbin}
\end{figure}


\begin{figure}
	\includegraphics[width=.32\textwidth]{Hval_SwitchOdd}
	\includegraphics[width=.32\textwidth]{Hbp_SwitchOdd}
	\includegraphics[width=.32\textwidth]{Hbpw_SwitchOdd}
	\includegraphics[width=.32\textwidth]{Cbp_SwitchOdd}
	\includegraphics[width=.32\textwidth]{Cbpw_SwitchOdd}
	\includegraphics[width=.32\textwidth]{MP_SwitchOdd}
	\includegraphics[width=.32\textwidth]{Period_SwitchOdd}
	\includegraphics[width=.32\textwidth]{HbpHval_SwitchOdd}
	\includegraphics[width=.32\textwidth]{HbpwHval_SwitchOdd}
	\includegraphics[width=.32\textwidth]{CbpHbp_SwitchOdd}
	\includegraphics[width=.32\textwidth]{CbpwHbpw_SwitchOdd}
	\caption{Statistical properties of EVEN, obtained by skipping the values in the odd position of the time series of  SWITCH,  using binary representation: (a) $H_{hist}$ vs $P$ (b) $H_{BP}$ vs $P$ (c) $C_{BP}$ vs $P$ (d) Number of missing ordering patterns $MP$ vs $P$. In Figures (a) to (d) dashed line correspond to floating point numbers. (e) representation in the $H_{hist},H_{BP}$ plane in the the binary numerical system.  The star represents the state for floating points numbers. (f) representation in the $H_{BP},C_{BP}$ plane.  The star represents the state for floating points numbers.  } \label{fig:seqimparbin}
\end{figure}