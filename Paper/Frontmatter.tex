\begin{frontmatter}

	\title{Complexity of switching chaotic maps}
	M. Antonelli$^{1,2}$, L. De Micco$^{1,2,3}$, O. A. Rosso$^{3,4,5,6}$ and H. A. Larrondo$^{1,2,3}$\\
	$^{1}$ Facultad de Ingenier\'ia, Universidad Nacional de Mar del Plata, Mar del Plata, Argentina.\\
	$^{2}$ ICyTE. Instituto de Investigaciones Científicas y Tecnológicas en Electrónica.\\
	$^{3}$ CONICET. Consejo Nacional de Investigaciones Científicas y Técnicas\\
	$^{4}$ Departamento de 
	Inform\'atica en Salud, Hospital Italiano de Buenos Aires,   C1199ABB Ciudad Autónoma de Buenos Aires, Argentina.\\
	$^{5}$ Instituto de F\'{\i}sica, Universidade Federal de Alagoas (UFAL), Macei\'o, Brazil.
	$^{6}$ Facultad de Ingeniería y Ciencias Aplicadas, Universidad de Los Andes, Santiago, Chile.
	

\begin{abstract}

In this paper we investigate the degradation of chaotic as consequence of their implementation in a digital media such as Digital Signal Processors (DSP), Field Programmable Gate Arrays (FPGA) or Application-Specific Integrated Circuits (ASIC).

In these systems, binary floating- and fixed-point are the numerical representations available.
Fixed-point representation is preferred over floating-point when speed, low power and/or small circuit area is necessary.
The specific period that every fixed-point precision produces was investigated in previous reports, using as example the tent map and the logistic map.
After this, authors applies switching and skipping techniques to enlarge the periods.

Statistical characteristics are also important.
It has been recently shown that it is convenient to describe the statistical characteristic using both, a causal and a non-causal quantifiers.
In this paper we complement the period analysis by characterizing the behaviour of these maps from an statistical point of view using causal and non-causal entropies and complexities.

\end{abstract}
\maketitle
\end{frontmatter}
{\bf VERSION: \today}