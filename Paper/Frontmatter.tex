\begin{frontmatter}

	\title{Complexity of switching chaotic maps}
	M. Antonelli$^{1}$, L. De Micco$^{1,2}$, O. A. Rosso$^{3,4}$ and H. A. Larrondo$^{1,2}$\\
	$^{1}$ Facultad de Ingenier\'ia, Universidad Nacional de Mar del Plata, Mar del Plata, Argentina.\\
	$^{2}$ CONICET.\\
	$^{3}$ LaCCAN/CPMAT Instituto de Computa\c{c}ao, Universidade Federal de Alagoas, Macei\'o,Alagoas, Brazil.\\
	$^{4}$ Laboratorio de Sistemas Complejos, Facultad de
	Ingenier\'ia, Universidad de Buenos Aires, Ciudad Aut\'onoma de
	Buenos Aires, Argentina.\\

\begin{abstract} 
%redactarlo de nuevo extractando lo mas importante y sin referencias, porque esta repetido de la introducci—n
In the last years digital measuring systems become the standard in
all experimental sciences because new programable electronic
devices, such as Digital Signal Processors ($DSP$) and Field
Programmable Gate Arrays ($FPGA$) allow experimenters to design and
modify their own measuring systems.

The effect of finite precision in these new devices needs to be
investigated, specially in the case of chaotic systems, because due to roundoff errors digital implementations will always become periodic with a period $T$. Furthermore floating point, decimal numbers with a finite number of digits and binary numbers are numerical representations available in new programmable devices and each of them produces specific statistical characteristics. It has been recently shown that it is convenient to describe the statistical characteristic using both, a non causal and a causal probability distribution function ($PDF$). The corresponding entropies, must be evaluated to quantify these $PDF$«s. The period $T$ needs  also to be evaluated. 

In this paper we study  two well
known chaotic maps: the tent map and the logistic map. All the above mentioned numerical representations are considered. Furthermore sequential switching between both maps is evaluated as a tool to improve the statistical characteristics. 

\end{abstract}
\maketitle
\end{frontmatter}
{\bf VERSION: \today}