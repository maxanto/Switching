\begin{frontmatter}

	\title{Complexity of switching chaotic maps}
	M. Antonelli$^{1}$, L. De Micco$^{1,2}$, O. A. Rosso$^{3,4}$ and H. A. Larrondo$^{1,2}$\\
	$^{1}$ Facultad de Ingenier\'ia, Universidad Nacional de Mar del Plata, Mar del Plata, Argentina.\\
	$^{2}$ CONICET.\\
	$^{3}$ LaCCAN/CPMAT Instituto de Computa\c{c}ao, Universidade Federal de Alagoas, Macei\'o,Alagoas, Brazil.\\
	$^{4}$ Laboratorio de Sistemas Complejos, Facultad de
	Ingenier\'ia, Universidad de Buenos Aires, Ciudad Aut\'onoma de
	Buenos Aires, Argentina.\\

\begin{abstract}
In the last years, digital systems such as Digital Signal Processors (DSP), Field Programmable Gate Arrays (FPGA) and Application-Specific Integrated Circuits (ASIC), became the standard in all experimental sciences.
Experimenters may design and modify their own systems.

In these systems digital implementations has a custom made numerical system, therefore finite arithmetic needs to be investigated.
Fixed point representation is prefered over floating point when  speed, low power and/or small circuit area is necessary.

Chaotic systems implemented in finite preceision will always became periodic with period T.
It has been recently shown that it is convenient to describe the statistical characteristic using both, a non causal and a causal probability distribution function ($PDF$).
The corresponding entropies, must be evaluated to quantify these $PDF$'s.  

Binary floating and fixed point are the numerical representations available, each of them produces specific period and statistical characteristics that needs to be evaluated.

In previous works a study about period was carried out using as an example two well known chaotic maps: the tent map and the logistic map. After that, switching techniques between these two maps were tested.
In this paper we characterize the behaviour of these maps from an statistical point of view using causal and non causal entropires.

\end{abstract}
\maketitle
\end{frontmatter}
{\bf VERSION: \today}