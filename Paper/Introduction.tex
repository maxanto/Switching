\section{Introduction} \label{sec:intro}

In the last years digital systems became the standard in all experimental sciences.
By using new programmable electronic devices such as DSP's, ASIC's or FPGA's, experimenters are allowed to design and modify their own signal generators, measuring systems, simulation models, etc.

Nowdays, the chaotic systems are widely used in digital electronics in fields such as electromagnetic compatibility, encrypted secure communications, controlled noise generators, etc \cite{Machado2004,Smaoui2009,DeMicco2017,Antonelli2012,DeMicco2007A,DeMicco2007B}.
These systems are specially interesting due to their extreme sensibility to initial conditions, nevertheless this characteristic is the source of the main difficulties at the time to implement them.  
In these fields the motivation of using chaotic systems is to genetrate sequences that meet certain requirements, rather than reproducing an exact replica of the real systems.

When a chaotic system is implemented in computers or any digital device, the chaotic attractor become periodic by the effect of finite precision, then only pseudo chaotic attractors can be generated \cite{Alcover2017,Dias2011}.
Discretization may even destroy the pseudo chaotic behavior and consequently is a non trivial process \cite{DeMicco2017,Azzaz2013}.

In these new devices, floating- and fixed-point are the \textcolor{red}{most common} arithmetics.
Floating-point is the more accurately solution but is not always recommended when speed, low power and/or small circuit area are required.
A fixed-point solution is better in these cases, nevertheless the feasibility of their implementation needs to be investigated.

The effect of numerical discretization over a chaotic map was recently addressed in \cite{DeMicco2017}, \cite{Nepomuceno2017}, \cite{Tlelo-Cuautle2016} and \cite{DelaFraga2017}.
In our previous work \cite{DeMicco2017} we have explored the statistical degradation of the phases' space for a family of 2D quadratic maps.
These maps present a multiatractor dynamic that makes them very attractive as random number generator in fields like criptography, encoding, etc.
Nepomuceno \textit{et al.} \cite{Nepomuceno2017} reported the existence of more than one pseudo-orbit of continuous chaotic systems when it is discretizated using different schemes.
\textcolor{red}{In \cite{Tlelo-Cuautle2016} and \cite{DelaFraga2017}, authors have proposed to use the value of the entropy to choose the number of bits in the fractional part, when maps are implemented in integer arithmetic}

Grebogi and coworkers \cite{Grebogi1988} saw that the period $T$ scales with roundoff $\epsilon$ as $T\sim\epsilon^{-d/2}$ where $d$ is the correlation dimension of the chaotic attractor.
This issue was also addressed in \cite{Persohn2012}, in this paper authors explore the evolution on the period $T$ as consequence of roundoff induced by 2-based numerical representation.
To have a large period $T$ is an important property of chaotic maps, in \cite{Nagaraj2008} Nagaraj \textit{et. al} studied the effect of switching over the average period lengths of chaotic maps in finite precision.
They saw that the period $T$ of the compound map obtained by switching between two chaotic maps is higher than the period of each map.
Liu \textit{et. al} \cite{Liu2006} studied different switching rules applied to linear systems to generate chaos.
Switching issue was also addressed in \cite{Gluskin2008}, the author considered some mathematical, physical and engineering aspects related to singular, mainly switching systems.
Switching systems naturally arise in power electronics and many other areas in digital electronics.
They have also interest in transport problems in deterministic ratchets \cite{Zarlenga2009} and it is known that synchronization of the switching procedure affects the output of the controlled system.
Chiou and Chen \cite{Chiou2009} published an analysis of the stabilization and switching laws to design switched discrete-time systems.
Recently, Borah \textit{et al.} \cite{Borah2017} presented a family of new chaotic systems and a switching based synchronization strategy.

Stochasticity and mixing are relevant to characterize a chaotic behavior.
To investigate these properties several quantifiers were studied \cite{DeMicco2009}.
Entropy and complexity from information theory were applied to give a measure for causal and non causal entropy and causal complexity.

A fundamental issue is the criterium to select the probability distribution function (PDF) assigned to the time series, causal and non causal options are possible.
Here we consider the non causal traditional PDF obtained by normalizing the histogram of the time series.
Its statistical quantifier is the normalized entropy $H_{hist}$ that is a measure of equiprobability among all allowed values.
We also considered a causal PDF that is obtained by assigning ordering patterns to segments of trajectory of length $D$.
This PDF was first proposed by Bandt \& Pompe in a seminal paper \cite{Bandt2002}, the corresponding entropy $H_{BP}$ was also proposed as a quantifier by Bandt \& Pompe.
In \cite{Rosso2007a} authors applied the causal complexity $C_{BP}$ to detect chaos.
Among them the use of an entropy-complexity representation ($H_{hist} \times C_{BP}$ plane) and causal-non causal entropy ($H_{BP} \times H_{hist}$ plane) deserves special consideration \cite{DeMicco2009,DeMicco2008,DeMicco2012,Rosso2007a,Rosso2010,Antonelli2017}.

Recently, amplitude information was introduced in \cite{Fadlallah2013} to add some immunity to weak noise in a causal PDF.
The new scheme better tracks abrupt changes in the signal and assigns less complexity to segments that exhibit regularity or are subject to noise effects.
Then, we define the causal entropy with amplitude contributions $H_{BPW}$ and the causal complexity with amplitude contributions $C_{BPW}$.
Also, we introduce the modified planes $H_{hist} \times C_{BP}$ and $H_{BP} \times H_{hist}$.

Amig\'o and coworkers proposed the number of forbidden patterns as a quantifier of chaos \cite{Amigo2007a}.
Essentially they reported the presence of forbidden patterns as an indicator of chaos.
Recently it was shown that the name forbidden patterns is not convenient and it was replaced by missing patterns (MP) \cite{Rosso2012}, in this work authors showed that there are chaotic systems that present MP from a certain minimum length of patterns.
Our main interest on MP is because it gives an upper bound for causal quantifiers.

Following \cite{Nagaraj2008}, in this paper we study the statistical characteristics of five maps, two well known maps: (1) the tent (TENT) and (2) logistic (LOG) maps, and three additional maps generated from them: (3) SWITCH, generated by switching between TENT and LOG; (4) EVEN, generated by skipping all the elements in odd positions of SWITCH time series and (5) ODD, generated by discarding all the elements in even positions of SWITCH time series.
Binary floating- and fixed-point numbers are used, these specific numerical systems may be implemented in modern programmable logic devices.

The main contribution of this paper is the study of how  different statistical quantifiers detect the evolution of stochasticity and mixing of the chaotic maps according as the numerical precision varies. 
To illustrate this sequences generated by well known maps were used, and also sequences obtained by randomization methods like skipping and switching.

Organization of the paper is as follows:
Section \ref{sec:quanti} describes the statistical quantifiers used in the paper and the relationship between their value and the characteristics of the causal and non causal PDF's considered;
Section \ref{sec:resultados} shows and discusses the results obtained for all the numerical representations.
Finally Section \ref{sec:conclusions} deals with final remarks and future work.