\section{Introduction} \label{sec:intro}

In the last years digital systems became the standard in all experimental sciences.
By using virtual instruments, new programmable electronic devices such as Digital Signal Processors (DSP) and Reconfigurable electronics such as Field Programmable Gate Arrays (FPGA) or Application-Specific Integrated Circuits (ASIC), allow experimenters to design and modify their own signal generators, measuring systems, simulation models, etc.

The effect of finite precision in these new devices needs to be investigated.
Floating-point is not always available when speed, low power and/or small circuit area are is required, a fixed-point solution is better in these cases.
Fixed-point representation is critical if chaotic systems must be implemented, because due to roundoff errors digital implementations will always become periodic with period $T$ and unstable orbits with a low periods may become stable destroying completely the chaotic behavior.

Grebogi and coworkers \cite{Grebogi1988} studied this subject and they saw that the period $T$ scales with roundoff $\epsilon$ as $T\sim\epsilon^{-d/2}$ where $d$ is the correlation dimension of the chaotic attractor.
To have a large period $T$ is an important property of chaotic maps, in \cite{Nagaraj2008} Nagaraj et. als studied the effect of switching over the average period lengths of chaotic maps in finite precision.
They saw that the period $T$ of the compound map obtained by switching between two chaotic maps is higher than the period of each map.
Liu et. als \cite{Liu2006} studied different switching rules applied to linear systems to generate chaos.
Switching issue was also addressed in \cite{Gluskin2008}, author considers some mathematical, physical and engineering aspects related to singular, mainly switching systems.
Switching systems naturally arise in power electronics and many other areas in digital electronics.
They have also interest in transport problems in deterministic ratchets \cite{Zarlenga2009} and it is known that synchronization of the switching procedure affects the output of the controlled system.

Stochasticity and mixing are also relevant, to characterize these properties several quantifiers were studied \cite{DeMicco2009}.
Among them the use of an entropy-complexity representation ($H-C$ plane) and causal-noncausal entropy ($H_{BP}-H_{Val}$ plane) deserves special consideration \cite{Rosso2007,DeMicco2008,DeMicco2012,DeMicco2009,Rosso2010,Antonelli2017}.
A fundamental issue is the criterium to select the probability distribution function (PDF) assigned to the time series.
Causal and non causal options are possible.
Here we consider the non-causal traditional PDF obtained by normalizing the histogram of the time series.
Its statistical quantifier is the normalized entropy $H_{Val}$ that is a measure of equiprobability among all allowed values.
We also consider a causal PDF that is obtained by assigning ordering patterns to segments of trajectory of length $D$.
This PDF was first proposed by Bandt \& Pompe in a seminal paper \cite{Bandt2002}.
The corresponding entropy $H_{BP}$ was also proposed as a quantifier by Bandt \& Pompe.
Amig\'o and coworkers proposed the number of forbidden patterns as a quantifier of chaos \cite{Amigo2007a}.
Essentialy they reported the presence of forbidden patterns as an indicator of chaos.
Recently it was shown that the name forbidden patterns is not convenient and it was replaced by missing patterns (MP) \cite{Rosso2012}.
\textcolor{red}{PONER ALGO DE BPW}



In this paper we study the statistical characteristics of five maps, two well known maps: (1) the tent map (TENT) and (2) logistic map (LOG), and three additional maps generated from them: (3) SWITCH, generated by switching between TENT and LOG; (4) EVEN, generated by skipping all the elements in odd position in SWITCH time series and (5) ODD, generated by discarding all the elements in an even position in SWITCH time series.
Binary floating- and fixed-point numbers are used, these specific numerical systems may be implemented in modern programmable logic devices.

The main contributions of this paper are:
\begin{enumerate}
\item the definition of different statistical quantifiers and their relationship with the properties of the series generated by the studied chaotic maps.
\item the study of how these quantifiers detect the evolution of stochasticity and mixing of the chaotic maps according as the numerical precision varies.
\item the effect on the period and the statistical properties of the time series of switching between two different maps.
\item the effect on the period and the statistical properties of the time series of skipping values in the switched maps.
\end{enumerate}

Organization of the paper is as follows:
section \ref{sec:quanti} describes the statistical quantifiers used in the paper and the relationship between their value and characteristics of the causal and non causal PDF considered;
section \ref{sec:resultados} shows and discusses the results obtained for all the numerical representations.
Finally section \ref{sec:conclusions} deals with final remarks and future work.