\subsubsection{Logistic map (LOG)} \label{subsubsec:log}

According as B grows, statistical properties vary until they stabilize.
Figs. \ref{fig:LOG_QuantiB} (a) to (f) show the statistical properties of LOG map in floating point and fixed point representation.
All these figures show: $100$ red points for each fixed point precision ($B$) and in black their average (dashed black line connecting black dots), $100$ horizontal dashed blue lines that are the results of each run in floating point and a black solid line their average.
In this case, all the lines of the floating point are overlapped.

For $B\geq 30$ the value of $H_{val}$ remains almost identical to the values for the floating point representation whereas $H_{BP}$ and $C_{BP}$ stabilizes at $B>21$.
Their values are: $<H_{val}>=0.9669$; $<H_{BP}>=0.6269$; $<C_{BP}>=0.4843$.
Note that the stable value of missing patterns $MP=645$ makes the optimum $H_{BP} \leq ln(75)/ln(720) \simeq 0.65$.
Then, $B=30$ is the most convenient choice because an increase in the number of fractional digits does not improve the statistical properties.

Some conclusions can be drawn regarding \textit{BP} and \textit{BPW} quantifiers.
For $B=1, 2, 3$ and $4$, the averaged $BP$ quantifiers are almost $0$ while the averaged $BPW$ quantifiers can not be calculated (seein Figs. \ref{fig:LOG_QuantiB} c and e the missing black dashed line).
For those sequences where the initial condition where $0$ all iterations result to be a sequence of $0$s (the fixed point of the map), this happens very likely with small precisions because of the roundoffs.

When $B$ increases, $B=7, 9$ and $12$, the used initial conditions are rounded to zero less frequently.
So the generated sequences start from some value but many of them fall to zero with a short transitory.
This can be seen in Figs. c and e, $BPW$ shows high dispersion unlike $BP$ quantifiers.
This is because $BPW$ procedure takes into account only the transient discarding fixed points, unlike $BP$ procedure that considers all the sequence. 

\begin{figure}
	\includegraphics[width= .49\textwidth]{Hval_Log}
	\includegraphics[width= .49\textwidth]{Hbp_Log}
	\includegraphics[width= .49\textwidth]{Hbpw_Log}
	\includegraphics[width= .49\textwidth]{Cbp_Log}
	\includegraphics[width= .49\textwidth]{Cbpw_Log}
	\includegraphics[width= .49\textwidth]{MP_Log}
	\caption{Statistical properties of the LOG map: (a) $H_{val}$ vs $B$ (b) $H_{BP}$ vs $B$ (c) $C_{BP}$ vs $B$ (d) MP vs $B$.}
	\label{fig:LOG_QuantiB}
\end{figure}

The same results are now shown in double entropy planes with the precision as parameter (Fig. \ref{fig:LOG_HH}).
These figures show: $100$ red points for each fixed point precision ($B$) and in black their average (dashed black line connecting black dots), $100$ blue dots that are the results of each run in floating point and a black star their average.
This last $100$ points and their average are overlapped.

As expected, the fixed point architecture implementation converges to the floating point value as $B$ increases.
For both, Hbp-Hval and Hbpw-Hval, from $B=20$, $H_{val}$ improves but $H_{BP}$ remains constant.
It can be seen that the distribution of values reaches high values ($<H_{val}>=0.9669$) but their mixing is poor ($<H_{BP}>=0.6269$).

\begin{figure}
	\includegraphics[width= .49\textwidth]{HbpHval_Log}
	\includegraphics[width= .49\textwidth]{HbpwHval_Log}
	\caption{Evolution of statistical properties in double entropy plane of LOG map: (a) $H_{val}$ vs $H_{BP}$ (b) $H_{val}$ vs $H_{BPW}$.}
	\label{fig:LOG_HH}
\end{figure}

In Fig. \ref{fig:LOG_HC} we show the entropy-complexity planes.
Dotted gray lines are the upper and lower margins, is expected that a chaotic system remains near the upper margin.
These results characterize a chaotic behaviour, in $H_{BP}-C_{BP}$ plane we can see a low entropy and high complexity.

\begin{figure}
	\includegraphics[width= .49\textwidth]{CbpHbp_Log}
	\includegraphics[width= .49\textwidth]{CbpwHbpw_Log}
	\caption{Evolution of statistical properties in entropy-complexity plane of LOG map: (a) $C_{BP}$ vs $H_{BP}$ (b) $C_{BPW}$ vs $H_{BPW}$.}
	\label{fig:LOG_HC}
\end{figure}
